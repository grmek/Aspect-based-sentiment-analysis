\documentclass[11pt,a4paper]{article}
\usepackage{times,latexsym}
\usepackage{url}
\usepackage[T1]{fontenc}

%% Package options:
%% Short version: "hyperref" and "submission" are the defaults.
%% More verbose version:
%% Most compact command to produce a submission version with hyperref enabled
%%    \usepackage[]{tacl2018v2}
%% Most compact command to produce a "camera-ready" version
%%    \usepackage[acceptedWithA]{tacl2018v2}
%% Most compact command to produce a double-spaced copy-editor's version
%%    \usepackage[acceptedWithA,copyedit]{tacl2018v2}
%
%% If you need to disable hyperref in any of the above settings (see Section
%% "LaTeX files") in the TACL instructions), add ",nohyperref" in the square
%% brackets. (The comma is a delimiter in case there are multiple options specified.)

\usepackage[acceptedWithA]{tacl2018v2}

\title{Aspect-based sentiment analysis}

% Author information does not appear in the pdf unless the "acceptedWithA" option is given
% See tacl2018v2.sty for other ways to format author information
\author{
    Miha Bizjak, Anže Gregorc, Rok Grmek \\
    University of Ljubljana \\
    Faculty for computer and information science \\
    Večna pot 113, SI-1000 Ljubljana \\
    mb9232@student.uni-lj.si, ag9497@student.uni-lj.si, rg6954@student.uni-lj.si
}

\date{}



\begin{document}



\maketitle



\begin{abstract}
    TODO
\end{abstract}



\section{Introduction}

Online news, forums and social media are a place for everyone to read and write articles and posts across various domains. 
People can also leave comments and giving their opinion and express their feelings about the topics. 
That leads  to a huge amount of text content. 
That is probably why natural language analysis is currently a hot topic around the world.
We wanted to extract useful information out of large amount of text data. Since we have no time for reading all the words that are written nowadays, we hope to build a good computer program to do that for us.
In this project we chose to do aspect-based sentiment analysis. 
Our task is to get the subjective information from text material that refer to a entity with the use of natural language processing and other methods. 
An entity is considered as a person, organization or a location and can be represented multiple times in one document or a sentence and there could be more entities in one document.

For the given task, we decided to test multiple approaches and develop different models for predicting the sentiment for each entity.
We will first define some really simple models as a starting point, and then we will try to derive some more complex models.
All of them will be targeting the Slovene language, and we will evaluate each of the models on the Slovene corpus for aspect-based sentiment analysis - SentiCoref 1.0~\cite{zitnik2019slovene}.



\section{Related work}

The main challenge of entity-based analysis is how to find words that describe the entity and identify if contributes to positive or negative sentiment to a given entity. 
A lot of related work tried to predict sentiment of the whole document. 
But in many cases a text can describe polarity of a more entities. 
That is why we suggest that sentiment analysis is done on entity level.
Since our task is more specific we focused more on methods that identify entities.

In the paper~\cite{ding2018entity} they developed an entity-based sentiment analysis SentiSW and tested it on issue comments from GitHub.
SentiSW can classify issue comments into \emph{<sentiment, entity>} tuples. 
They evaluate the entity recognition by manually annotation and it achieves 75\% accuracy. 
The main pipeline of this tool is preprocess (words removal, words replacing, stem), feature vectorize (TF-IDF, Doc2vec), classifier (random forest, bagging and other supervised machine learning methods) and entity recognition (rule-based method). 

The use of word embeddings provide powerful methods for semantic understanding without the need of creating large amounts of annotated test data. 
The paper~\cite{sweeney2017multi} enhanced the word embeddings approach with the  deployment of a sentiment lexicon-based technique to appoint a total score that indicates the polarity of opinion in relation to a particular entity. 
They associate a given entity with the words describing it and extracting the associated sentiment to try to infer if the text is positive or negative in relation to the entity.



\section{Methods}

\subsection{Dataset}

TODO

\subsection{Models}

TODO

\subsubsection{Random model}

TODO
\begin{itemize}
    \item Randomly assigns sentiment from 1 to 5 to each entity.
    \item No real value.
    \item Developed along with the evaluation toolbox only for testing.
    \item Also serves as a model that should be evaluated with the lowest possible score.
\end{itemize}

\subsubsection{Majority model}

TODO
\begin{itemize}
    \item Assigns the neutral sentiment (3) to all entities.
    \item Should produce a decent score because of the distribution of the sentiment classes.
    \item Will serve for a reference score - complex models should not be performing worse than this simple majority model.
\end{itemize}

\subsubsection{Lexicon Features model}

TODO
\begin{itemize}
    \item ...
\end{itemize}

\subsubsection{Further ideas}

TODO (note: following ideas will be used for implementing a few more models)
\begin{itemize}
    \item Current features in the Lex. Feat. model depend mainly on the positive/negative words in the neighborhood of the entity. Instead of looking at the neighborhood, we could use a dependency parser and observe sentiment of the most related words (not necessary in the neighbourhood).
    \item Instead of constructing handcrafted features, we could use BERT model for feature extraction. Those features depend on the context of each word, so we could simply use feature representation of each entity word occurrence and its sentiment as a learning sample for some classifier. With the trained classifier, we could predict sentiment for each occurrence of the entity and calculate the average sentiment for the entity.
    \item Test the effect of using different classifiers (re-implement a single model with different classifiers)
    \item Combine multiple feat. representations into a single model (normalize and concatenate feature vectors from different models and let the classifier use/learn most important features from all models).
\end{itemize}

\subsection{Evaluation}

TODO
\begin{itemize}
    \item Describe the train/test split.
    \item Implemented measures: Accuracy, Precision, Recall, F1 score.
\end{itemize}



\section{Results}

TODO



\section{Discussion}

TODO



\section{References}

\bibliography{bibliography}
\bibliographystyle{acl_natbib}



\end{document}
